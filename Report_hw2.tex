%%%%%%%%%%%%%%%%% DO NOT CHANGE HERE %%%%%%%%%%%%%%%%%%%% 
%%%%%%%%%%%%%%%%%%%%%%%%%%%%%%%%%%%%%%%%%%%%%%%%%%%%%%%%%%{
    \documentclass[twoside,11pt]{article}
    %%%%% PACKAGES %%%%%%
    \usepackage{pgm2016}
    \usepackage{amsmath}
    \usepackage{algorithm}
    \usepackage[noend]{algpseudocode}
    \usepackage{subcaption}
    \usepackage[english]{babel}	
    \usepackage{paralist}	
    \usepackage[lowtilde]{url}
    \usepackage{fixltx2e}
    \usepackage{listings}
    \usepackage{color}
    \usepackage{hyperref}
    
    \usepackage{auto-pst-pdf}
    \usepackage{pst-all}
    \usepackage{pstricks-add}
    
    %%%%% MACROS %%%%%%
    \algrenewcommand\Return{\State \algorithmicreturn{} }
    \algnewcommand{\LineComment}[1]{\State \(\triangleright\) #1}
    \renewcommand{\thesubfigure}{\roman{subfigure}}
    \definecolor{codegreen}{rgb}{0,0.6,0}
    \definecolor{codegray}{rgb}{0.5,0.5,0.5}
    \definecolor{codepurple}{rgb}{0.58,0,0.82}
    \definecolor{backcolour}{rgb}{0.95,0.95,0.92}
    \lstdefinestyle{mystyle}{
       backgroundcolor=\color{backcolour},  
       commentstyle=\color{codegreen},
       keywordstyle=\color{magenta},
       numberstyle=\tiny\color{codegray},
       stringstyle=\color{codepurple},
       basicstyle=\footnotesize,
       breakatwhitespace=false,        
       breaklines=true,                
       captionpos=b,                    
       keepspaces=true,                
       numbers=left,                    
       numbersep=5pt,                  
       showspaces=false,                
       showstringspaces=false,
       showtabs=false,                  
       tabsize=2
    }
    \lstset{style=mystyle}
%%%%%%%%%%%%%%%%%%%%%%%%%%%%%%%%%%%%%%%%%%%%%%%%%%%%%%%%%% 
%%%%%%%%%%%%%%%%%%%%%%%%%%%%%%%%%%%%%%%%%%%%%%%%%%%%%%%%%% }

%%%%%%%%%%%%%%%%%%%%%%%% CHANGE HERE %%%%%%%%%%%%%%%%%%%% 
%%%%%%%%%%%%%%%%%%%%%%%%%%%%%%%%%%%%%%%%%%%%%%%%%%%%%%%%%% {
\newcommand\course{CSE 331}
\newcommand\courseName{Computer Organization}
\newcommand\studentName{Elif Akgün}                  % <-- YOUR NAME
\newcommand\studentNumber{1801042251}                % <-- STUDENT ID #
%%%%%%%%%%%%%%%%%%%%%%%%%%%%%%%%%%%%%%%%%%%%%%%%%%%%%%%%%% }
%%%%%%%%%%%%%%%%%%%%%%%%%%%%%%%%%%%%%%%%%%%%%%%%%%%%%%%%%%

%%%%%%%%%%%%%%%%% DO NOT CHANGE HERE %%%%%%%%%%%%%%%%%%%% 
%%%%%%%%%%%%%%%%%%%%%%%%%%%%%%%%%%%%%%%%%%%%%%%%%%%%%%%%%%
%{

    \ShortHeadings{Gebze Technical University -  \course ~~ \courseName}{\studentName - \studentNumber}
    \firstpageno{1}
    
    \begin{document}
    
    \title{Homework 2 Report\assignmentNumber}
    
    \author{\name \studentName \email \studentEmail \\
    \studentNumber
    \addr
    }
    
    \maketitle
%%%%%%%%%%%%%%%%%%%%%%%%%%%%%%%%%%%%%%%%%%%%%%%%%%%%%%%%%%
%%%%%%%%%%%%%%%%%%%%%%%%%%%%%%%%%%%%%%%%%%%%%%%%%%%%%%%%%% }

\section{Design Explanation} 
\label{sec:background}

In this program the task is to find if a subset of array elements can sum up to the target num. If not possible the output will be “Not possible!”. If it is possible, output is “Possible!”. Every array element can use only once.and only positive integers are allowed as array elements. Finding only one combination is enough to output “Possible!”. \newline
As solution, I used the backtracking technique.The backtracking technique generates every subset only once. There are two options for every element in the set, either subset includes this element or it does not. I set the maximum size of the array to 100. So the size of a subset can be  maximum 100. That's why, I created the same size \emph{subset} array to store subset elements. There is a flag to notify target sum is got. Also there is an variable \emph{k} which keeps current size of subset array and its current index. In for loop, kth index of subset array becomes array's ith element. Then with recursive call, it moves to next element of array. After recursive call, it checks sum of subset array's elements. \emph{sum} variable keeps sum of the each subset's elements. If sum equals target sum, then flag becomes 1 to notify expected sum got. If sum does not equal to target sum, k decreases 1 for backtracking and it goes on like this. When target sum is found, then recursive calls don't enter for loop and return. Finally, when target sum is found, \emph{CheckSumPossibility} function returns 1, otherwise it returns 0. \newline
Also I did bonus part. I kept related subset's elements in the subset array. Then I printed it to console after "Subset: " message.

\section{Usage} 
\label{sec:background}

In both parts, to run program, you should enter array's size after "Please enter the size of the array:" message. Then, you should enter the target sum after "Please enter the target sum:" message. Finally, you should enter the array elements after "Please enter the elements of array:" message. You can enter as many elements as the size of array. After enter elements, result shoul be come to console. I set the maximum size of the array to 100 for both part. You should not enter a number bigger than 100 for size.

\newpage

\section{Tests} 
\label{sec:background}

\subsection{Part 1} 
\label{sec:background}

\begin{figure}[h]
\caption{C++ Test 1}
\centering
\includegraphics[width=1\textwidth]{figure/c1.png}
\end{figure}

\begin{figure}[h]
\caption{C++ Test 2}
\centering
\includegraphics[width=1\textwidth]{figure/c2.png}
\end{figure}

\begin{figure}[h]
\caption{C++ Test 3}
\centering
\includegraphics[width=1\textwidth]{figure/c3.png}
\end{figure}

\begin{figure}[h]
\caption{C++ Test 4}
\centering
\includegraphics[width=1\textwidth]{figure/c4.png}
\end{figure}

\begin{figure}[h]
\caption{C++ Test 5}
\centering
\includegraphics[width=1\textwidth]{figure/c5.png}
\end{figure}

\begin{figure}[h]
\caption{C++ Test 6}
\centering
\includegraphics[width=1\textwidth]{figure/c6.png}
\end{figure} 

\newpage

\subsection{Part 2} 
\label{sec:background}

\textbf{MIPS Assembly Test 1:} \newline
Please enter the size of the array: 8 \newline
Please enter the target sum: 129 \newline
Please enter the elements of array: \newline
92 82 21 16 18 95 47 26

\begin{figure}[h]
\caption{MIPS Assembly Test 1}
\centering
\includegraphics[width=1\textwidth]{figure/m1.png}
\end{figure}

\textbf{MIPS Assembly Test 2:} \newline
Please enter the size of the array: 8 \newline
Please enter the target sum: 129 \newline
Please enter the elements of array: \newline
71 38 69 12 67 99 35 94 \newline

\begin{figure}[h]
\caption{MIPS Assembly Test 2}
\centering
\includegraphics[width=1\textwidth]{figure/m2.png}
\end{figure}

\newpage

\textbf{MIPS Assembly Test 3:} \newline
Please enter the size of the array: 8 \newline
Please enter the target sum: 129 \newline
Please enter the elements of array: \newline
95 42 27 36 91 4 2 53 \newline

\begin{figure}[h]
\caption{MIPS Assembly Test 3}
\centering
\includegraphics[width=1\textwidth]{figure/m3.png}
\end{figure}

\textbf{MIPS Assembly Test 4:} \newline
Please enter the size of the array: 9 \newline
Please enter the target sum: 132 \newline
Please enter the elements of array: \newline
25 32 48 49 3 51 68 73 8 \newline

\begin{figure}[h]
\caption{MIPS Assembly Test 4}
\centering
\includegraphics[width=1\textwidth]{figure/m4.png}
\end{figure}

\newpage

\textbf{MIPS Assembly Test 5:} \newline
Please enter the size of the array: 9 \newline
Please enter the target sum: 300 \newline
Please enter the elements of array: \newline
25 32 48 49 3 51 68 73 8 \newline

\begin{figure}[h]
\caption{MIPS Assembly Test 5}
\centering
\includegraphics[width=1\textwidth]{figure/m5.png}
\end{figure}

\textbf{MIPS Assembly Test 6:} \newline
Please enter the size of the array: 7 \newline
Please enter the target sum: 153 \newline
Please enter the elements of array: \newline
10 32 2 8 60 1 9 \newline

\begin{figure}[h]
\caption{MIPS Assembly Test 6}
\centering
\includegraphics[width=1\textwidth]{figure/m6.png}
\end{figure}

\end{document}
